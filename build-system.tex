
\documentclass[12pt,a4paper]{article}
\usepackage[french,english]{babel}
\usepackage{alertmessage}
\usepackage{twemojis}
\usepackage[bottom]{footmisc}
\usepackage[margin=1in]{geometry}
\usepackage{noweb}
\usepackage{pygmentex}
\usepackage{tcolorbox}
\usepackage[colorlinks=true,pdftitle=gb-starter-kit]{hyperref}

\def\nwbegincode#1{%
  \vskip5pt\noindent}
\def\nwendcode{}
\let\nwdocspar=\smallbreak

\author{\href{https://eldred.fr}{Eldred \bsc{Habert}}\thanks{Contributions by JL2210, Damian Yerrick, Evie, and Starleaf.}}
\title{\texttt{gb-starter-kit} build system breakdown}

\begin{document}
\maketitle
\renewcommand{\abstractname}{Summary and Intended Audience}
\begin{abstract}
	This document is useful for people who wish to understand why gb-starter-kit's Makefile (build script) is written in the way that it is, why and how it does what it does, and perhaps also gain some insight into Makefile good practices.

	It does contain a fairly quick primer on what a Makefile \emph{is}, so it is intended to be suitable to people who have never touched a Makefile before.
\end{abstract}
\tableofcontents

\pagebreak

\input{Makefile}

\end{document}
